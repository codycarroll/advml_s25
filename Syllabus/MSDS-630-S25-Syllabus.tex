\documentclass[11pt]{article}
\usepackage{amsmath,underscore,amssymb,amsfonts,amsthm,array,hhline,setspace,graphicx,url,verbatim}

\oddsidemargin -0.1in
\evensidemargin 0.0in
\textwidth 6.5in
\headheight 0.0in
\topmargin -0.5in
\textheight 9.7in

\setstretch{1.0}

\newenvironment{itemize*}%
  {\begin{itemize}%
    \setlength{\itemsep}{0pt}%
    \setlength{\parskip}{0pt}}%
  {\end{itemize}}

\begin{document}
\pagestyle{empty}

\noindent \begin{center}
\begin{Large}\textbf{MSDS 630 -- Advanced Machine Learning}\\
\textbf{Instructor: Cody Carroll}\\
\textbf{Course Syllabus}\\
\textbf{Spring 2025}\\\end{Large}
\end{center}

\vspace{0.05in}

\fbox{
\parbox{15cm}{
\begin{center}
\noindent \textbf{SUMMARY INFORMATION}\\
\noindent \textbf{Office:} 101 Howard, \#525 \\
\noindent \textbf{Email Address:} \url{cjcarroll@usfca.edu}\\\vspace{0.10in}
\noindent \textbf{Class Time:}  TR 10a-12p or 1p-3p in SFD 529
\noindent \textbf{Office Hours:} Tuesdays 3:30p-4:30p in person and  Mondays 1p-2p on [Zoom](https://usfca.zoom.us/my/cody.carroll) 
(subject to change during quiz weeks - watch the course slack!)
Zoom link: \url{https://usfca.zoom.us/my/cody.carroll}
\end{center}
}
}

\vspace{0.05in}

\noindent \textbf{Course Learning Objectives.}\\
On completion of this course the student should be able to:
\vspace{-0.05in}

\begin{itemize*}
\item Describe and apply selected learning algorithms / models and their variants (see Course Content below).
\item Select the appropriate learning algorithm or approach for a given situation or dataset.
\item Implement machine learning algorithms from scratch in Python.
\item Implement basic Neural Networks in PyTorch.
\item Implement advanced feature engineering techniques.
\end{itemize*}

\noindent As part of a team, the student will also carry out a machine learning project from start to finish, including:
\begin{itemize*}
\item Researching literature related to the problem.
\item Preparing the data for application of algorithms, including feature engineering.
\item Choosing and applying  appropriate ML algorithms (as well as hyper-parameter tuning).
\item Evaluating and communicating results both orally and in writing.
\end{itemize*}


\noindent \textbf{Course Content.} 

\vspace{-0.05in}

\begin{itemize*}
\item Dimension Reduction: SVD and PCA
\item Recommendation Systems: Collaborative Filtering, Matrix Factorization.
\item Boosting: Adaboost, Gradient Boosting.
\item Neural Networks, Pytorch.
\item Applications of Neural Networks.
\end{itemize*}



\noindent \noindent \textbf{Textbooks.} \\
\vspace{-0.15in}
\begin{enumerate}
\item Pattern Recognition and Machine Learning. Bishop. 
\item The Elements of Statistical Learning: Data Mining, Inference, and Prediction. Trevor Hastie, Robert Tibshirani, Jerome Friedman. \url{https://statweb.stanford.edu/~tibs/ElemStatLearn/} 
\item Deep Learning. Ian Goodfellow and Yoshua Bengio and Aaron Courville.  \url{http://www.deeplearningbook.org/}
\item Mining of Massive Datasets. Jure Leskovec, Anand Rajaraman and Jeffrey D. Ullman (Chapters 8, 9). \url{www.mmds.org}
\item  Probabilistic Machine Learning. Kevin Murphy.
\end{enumerate}

\noindent \textbf{Grading.} Part of my job as an instructor is to assign grades fairly and in a manner that reflects the high academic standards at the University of San Francisco and in the MSDS program. Your grade in this course will be computed according to the following weights:\\

\vspace{-0.15in}

\begin{center}
	\begin{tabular}{|c|c|}
		\hline
		\textbf{Component} & \textbf{Weight} \\
		\hline
		Homework & 20\%  \\
		Exams & 50\%  \\
		Project & 25\% \\
		Attendance/Professionalism & 5\%\\
		\hline
	\end{tabular}
\end{center}

\noindent \textbf{Homework.}

\begin{itemize}
	\item You will be assigned computational and theoretical homework assignments to be completed and turned in on Github every \textit{Friday} \textbf{before midnight} at \textit{11:59p Pacific} time, with a 48 hour grace period.  Github repositories will lock after this grace period ends. 

\item Students are encouraged to discuss and work together on assignments, but each student must turn in their own original work. \textbf{\textit{If there is evidence that the work turned in is not original work, which includes copying another student’s homework or using any solutions found online, all credit for that homework set will be forfeited. Homework is not to be posted to online help sites. These sites will be checked frequently.}}

\item \textbf{\textit{No late homework past the grace period will be accepted.}}

\end{itemize}


\noindent \textbf{Exams.} 

\begin{itemize}
\item You will be required to complete 2 quizzes, tentatively scheduled for: Thurs, Sept 12 and Thurs, Oct 3.  All quizzes will be closed book. Details will be discussed in classes leading up to quiz dates.

\item \textbf{In order to pass this class, your average quiz grade for this section must be at least 60\%.}

\item No make-up or early quizzes will be given in order to ensure fairness and integrity of the class. Missing an exam without proper documentation of a personal illness or family emergency will result in a score of zero for that exam. Any documentation must be submitted to the instructor before the exam in question at the earliest possible date.
\end{itemize}

\noindent \textbf{Project.} \\
Project details will be announced in class. \\

\noindent \textbf{Attendance and Professionalism. }

\begin{itemize}
\item Attendance is expected in all live lectures. Valid excuses for absence with permission will be accepted with documentation, but students are required to watch the lecture videos and submit class exercises and activities on Canvas/Github if any. Students who miss the live lectures with a valid excuse are required to submit the exercises within 24 hours of class time (3pm PST/PDT next day).

\item Professional behavior is expected both during classtime and outside of class when working with your fellow students. Issues with unprofessional behavior will result in deductions in your professionalism score.
\end{itemize}

\newpage

\noindent \textbf{On grades.} The MSDS program considers a grade of ``A" to represent exceptional work with respect to both the instructor's expectations and peer student achievements. A grade of ``B" represents the expected outcome, what is called ``competence" in a business setting. A ``C" grade represents achievements lower than the instructor's expectations for competence in the subject. A grade of ``F" represents unacceptably low level of knowledge and understanding of subject matter.  Scores less than 60\% on Exams or less than 60\% on the overall grade are considered ``F" in this class. \\

\noindent \textbf{On cheating.} As a Jesuit institution committed to \emph{cura personalis}---the care and education of the whole person---the University of San Francisco has an obligation to embody and foster the values of honesty and integrity. The university upholds standards of honesty and integrity from all members of the academic community, including faculty, students, and staff. All students are expected to know and to adhere to the university's honor code. You can find the full text of the code online at \url{http://www.usfca.edu/catalog/policies/honor/}. You are also bound by the terms of the MSDS Code of Conduct that you signed prior to matriculating in the analytics program. Refer to ON HOMEWORK sections for details regarding student collaboration on each category of deliverable. Plagiarism consists of copying \emph{any} material from \emph{any} source and submitting it as your own original work, regardless of where that material was sourced: the Internet, a book, textbook, or from deliverables previously submitted by other students. All students involved in any cheating or plagiarized deliverables, i.e., the cheater as well as the person(s) who willfully enabled or facilitated the act of cheating, will be reported to the MSDS Program Director. If you ever have questions about what constitutes plagiarism, cheating, or academic dishonesty in this course, I am happy to discuss with you at your convenience.\\

\noindent \textbf{On disability.} If you are a student with a disability or disabling condition, or if you think you may have a disability, please contact USF Student Disability Services (SDS) at 415.422.2613 within the first week of class, or immediately upon onset of the disability, to speak with a disability specialist. If you are determined eligible for reasonable accommodations, please meet with your disability specialist so they can arrange to have your accommodation letter sent to me, and we will discuss your needs for this course. For more information, please visit \url{http://www.usfca.edu/sds/} or call 415.422.2613. {\bf Accommodations are not retroactive.}\\

\noindent \textbf{On behavioral expectations.} All students are expected to behave in accordance with the Student Conduct Code and University policies (see \url{http://www.usfca.edu/fogcutter/}).  Open discussion and disagreement is encouraged when done respectfully and in the spirit of academic discourse. There are also a variety of behaviors that, while not against a specific University policy, may create disruption in this course. Students whose behavior is disruptive or who fail to comply with the instructor may be dismissed from the class for the remainder of the class period and may need to meet with the instructor or Dean prior to returning to the next class period. If necessary, referrals may also be made to the Student Conduct process for violations of the Student Conduct Code. \\

\noindent \textbf{On the learning \& writing center.} The Learning \& Writing Center provides assistance to all USF students in pursuit of academic success. Peer tutors provide regular review and practice of course materials in the subjects of Math, Science, Business, Economics, Nursing and Languages. Other content areas can be made available by student request. To schedule an appointment, log on to TutorTrac at \url{https://tutortrac.usfca.edu}. Students may also take advantage of writing support provided by Rhetoric and Language Department instructors and academic study skills support provided by Learning Center professional staff. For more information about these services contact the Learning \& Writing Center at 415.422.6713, \verb!lwc@usfca.edu! or stop by Cowell 215. Information may also be found at \url{twww.usfca.edu/lwc}.\\

\noindent \textbf{On Counseling and Psychological Services.}  Our diverse staff offers individual, couple, and group counseling to student members of our community. Services are confidential and free of charge. Call 415.422.6352 for an initial consultation appointment. Having a crisis at 3 AM? We are still here for you. Telephone consultation after hours is available between the hours of 5:00 PM to 8:30 AM; call the above number and press 2.\\

\noindent \textbf{On Illnesses and Emergencies.} If you fall ill or have an emergency (personal or otherwise) that significantly affects your ability to complete a project or take an exam, you must notify the instructor before the task or artifact is due. Do not simply skip an exam or an assignment and say you were sick after the fact. Always make arrangements with the instructor beforehand, rather than declaring illness or emergency later. **Accommodations are not retroactive.**  Illness and emergency related situations must be disclosed to both the instructor and program director in writing. Illness-related issues must be accompanied by a doctor’s note.\\

\noindent \textbf{On confidentiality, mandatory reporting and sexual assault.} As an instructor, one of my responsibilities is to help create a safe learning environment on our campus. I also have a mandatory reporting responsibility related to my role as a faculty member. I am required to share information regarding sexual misconduct or information about a crime that may have occurred on USF's campus with the University. Here are other resources:
\begin{itemize}
\item To report any sexual misconduct, students may visit Anna Bartkowski (UC 5$^{th}$ floor) or see many other options by visiting our website: \url{www.usfca.edu/student_life/safer}
\item Students may speak to someone confidentially, or report a sexual assault confidentially by contacting Counseling and Psychological Services at 415.422.6352. 
\item To find out more about reporting a sexual assault at USF, visit USF's Callisto website at: \url{www.usfca.callistocampus.org}.
\item For an off-campus resource, contact San Francisco Women Against Rape 415.647.7273 (\url{www.sfwar.org}).
\end{itemize}


\end{document}